%!TEX program = xelatex
\documentclass[hyperref,UTF8,12pt, twoside]{sdnubachelor}

\begin{document}
	
%---信息部分-------------------------------------------------------------------------------
	\sdnutitle{糯米排骨制作分析与算法研究}{English Title}
	\sdnuauthor{王小豆}{Xiaodou Wang}{201711010200}
	\sdnumentor{大脑袋}
	\sdnuinfo{信息科学与工程学院}{计算机科学与技术专业}{计工本1702}
	\sdnudate{2021}{4}{15}
	\sdnukeyw{关键字1;关键字2;关键字3;关键字4}{key1, key2, key3, key4}
	
	% 文档格式设置
	\sdnupaperprint{0}  % 是否为打印版本,打印版会在部分页面后加空页(判断奇偶),满足双面打印要求,1代表打印版本,0代表电子版
	\sdnusectioncenter{1}  % 标题是否居中,1代表居中,0代表居左
	
%-----封面部分(无需封面请注释\include一行)-----
	{
		\newlength{\nlength}
		\setlength{\nlength}{10cm}	%封面题目线长度,根据题目长短自行调整,默认10cm
		\begin{titlepage}
	\vspace*{3mm}
	\begin{center}
		\includegraphics[width=0.77\textwidth,trim=8 0 0 0,clip]{data/resource/logo.jpg}
	\end{center}
	\vspace*{1cm}
	\fontsize{50pt}{24pt}
	\centering
	\textbf{本}\hfill\textbf{科}\hfill\textbf{毕}\hfill\textbf{业}\hfill\textbf{论}\hfill\textbf{文}
	\\
	\vspace*{7.7cm}
	\zihao{3}
	%\justifying
	题~ \quad ~目:\underline{\makebox[\nlength]{\sdnutitlechs}}\\\vspace*{1mm}
	姓~ \quad ~名:\underline{\makebox[\nlength]{\sdnuauthorchs}}\\\vspace*{1mm}
	学~ \quad ~号:\underline{\makebox[\nlength]{\sdnuauthorid}}\\\vspace*{1mm}
	专~ \quad ~业:\underline{\makebox[\nlength]{\sdnumajorchs}}\\\vspace*{1mm}
	指导教师:\underline{\makebox[\nlength]{\sdnumentorchs}}\\\vspace*{1mm}
	学院(部):\underline{\makebox[\nlength]{\sdnucollegechs}}\\\vspace*{2.2cm}
	\sdnuyear 年\sdnumon 月\sdnuday 日
\end{titlepage}
%\renewcommand\paperprint[1]{
%	\ifnum\@sdnupaperprint=1 {\pagestyle{empty}
%		~\newpage}
%	\else {}
%	\fi
%}
		\ifnum\sdnuprint=1 {\pagestyle{empty}~\newpage}
		\fi
	}

	
%-----声明部分(无需声明请注释\include一行)-----
	{
		\begin{titlepage}
	\vspace*{1.3cm}
	\begin{center}
	\fontsize{16pt}{\baselineskip}
	\textbf{独}\quad\textbf{创}\quad\textbf{声}\quad\textbf{明}
	\end{center}
	\vspace*{24pt}
	\begin{spacing}{1.5}
	\fontsize{12pt}{\baselineskip}
	本人声明所呈交的学位论文是本人在导师指导下进行的研究工作及取得的研究成果。据我所知,除了文中特别加以标注和致谢的地方外,论文中不包含其他人已经发表或撰写过的研究成果,也不包含为获得\underline{\quad\quad\quad\quad\quad\quad\quad\quad}(注:如没有其他需要特别声明的,本栏可空)或其他教育机构的学位或证书使用过的材料。与我一同工作的同志对本研究所做的任何贡献均已在论文中作了明确的说明并表示谢意。
	\end{spacing}
	\vspace*{18pt}
	\begin{flushright}
	学位论文作者签名:\quad\quad\quad\quad\quad\quad\quad\quad 导师签字: \quad\quad\quad\quad\quad\quad\quad\quad
	\end{flushright}
	\vspace*{54pt}
	\begin{center}
	\fontsize{16pt}{\baselineskip}
	\textbf{学}\quad\textbf{位}\quad\textbf{论}\quad\textbf{文}\quad\textbf{版}\quad\textbf{权}\quad\textbf{使}\quad\textbf{用}\quad\textbf{授}\quad\textbf{权}\quad\textbf{书}
	\end{center}
	\vspace*{18pt}
	\begin{spacing}{1.5}
	\fontsize{12pt}{\baselineskip}
	本学位论文作者完全了解\underline{~~\textbf{学校}~~}有关保留、使用学位论文的规定,有权保留并向国家有关部门或机构送交论文的复印件和磁盘,允许论文被查阅和借阅。本人授权\underline{~~\textbf{学校}~~}可以将学位论文的全部或部分内容编入有关数据库进行检索,可以采用影印、缩印或扫描等复制手段保存、汇编学位论文。(保密的学位论文在解密后适用本授权书)
	\end{spacing}
	\begin{flushright}
	学位论文作者签名:~~\quad\quad\quad\quad\quad\quad\quad\quad\quad\quad~导师签字:~\quad\quad\quad\quad\quad\quad\quad\quad\quad\quad
	\vspace*{18pt}

	签字日期:20\quad 年\quad 月\quad 日\quad\quad\quad\quad\quad\quad\quad\quad 签字日期:20\quad 年\quad 月\quad 日 \quad\quad\quad\quad\
	\end{flushright}
	\vspace*{18pt}
	
\end{titlepage}
		\ifnum\sdnuprint=1 {\pagestyle{empty}~\newpage}
		\fi
	}

%-----目录生成部分-----
	{
		\begin{titlepage}
			\pagestyle{empty}
			\tableofcontents
			\label{content}
		\end{titlepage}
		\ifthenelse{\sdnuprint=1}{\ifthenelse{\isodd{\pageref{content}}}{\pagestyle{empty}~\newpage}{}}{}
	}

%-----摘要部分(编辑摘要请前往./data/resource/abstract.tex,无需摘要请注释\include一行)-----
{
	\sdnuinfoen{Shandong Normal University}{School of Information Science and Engineering}
	\sdnuclassification{TP393} %中图分类号
	\pagenumbering{roman}  % Alph(alph) / Roman(roman)  摘要页码形式,字母或者罗马数字,看个人喜好
	\clearpage
\setcounter{page}{1}
\begin{center}
\fontsize{15.75pt}{\baselineskip}
\vspace*{24pt}
\heiti{\sdnutitlechs}  % 题目-黑体三号

\vspace*{14pt}
\fontsize{14pt}{\baselineskip}
\fangsong{\sdnuauthorchs} % 作者-仿宋四号

% \vspace*{20pt}
\fontsize{10.5pt}{20pt}
\fangsong{(山东师范大学\sdnucollegechs\sdnumajorchs\sdnuclasschs )} % 学校-仿宋五号
\vspace*{18pt}

\end{center}

% 中文摘要
\phantomsection
\addcontentsline{toc}{section}{摘要}
% \section*{摘要}
% \vspace*{-2mm}

% \setlength{\baselineskip}{25pt}
\fontsize{12pt}{25pt}
\textbf{摘要:}中文摘要与关键词:对毕业论文(设计)核心内容的集中陈述,简要说明研究目的、研究方法、主要结果(结论),150-300字,中文摘要后需列出3-5个关键词。(中英文摘要合起来尽量独占一页,如果内容较多也可占据两页)

\vspace*{14pt}
\textbf{关键词:}\sdnukeywchs

\vspace*{4pt}
\fontsize{10.5pt}{25pt}
中图分类号:\textrm{\classification}
\vspace*{18pt}

% 英文摘要
% \newpage
\phantomsection
\addcontentsline{toc}{section}{Abstract}
% \section*{Abstract}
% \vspace*{-2mm}

\begin{center}
\fontsize{15.75pt}{\baselineskip}
\vspace*{24pt}
\textrm{\textbf{\sdnutitleen}}  % 题目-新罗马三号

\fontsize{12pt}{\baselineskip}
\vspace*{12pt}
\textrm{\sdnuauthorens} % 作者-新罗马小四

\textrm{(\sdnucollegeen, \sdnuen)} % 单位-新罗马小四

\end{center}

\fontsize{12pt}{25pt}
\noindent\textbf{Abstract:~}The content of the English abstract should be basically the same as that of the Chinese abstract, and the English keywords corresponding to the Chinese keywords should be listed after the English abstract.

\vspace*{6pt}
\textbf{Key words:~}\sdnukeywen
\label{abstract}

	
	\ifthenelse{\sdnuprint=1}{\ifthenelse{\equal{\pageref{abstract}}{A}}{\pagestyle{empty}~\newpage}{}}{}
	
%	\pageref{unknown}=A {\pagestyle{empty}~\newpage}

	
}

%-----正文部分,编辑正文请前往./data/resource/article.tex-----
%\ifnum\sdnuseccenter=0 {\titleformat{\section}{\raggedright\Large\bfseries}{\thesection .\quad}{0pt}{}}
%\else 
%\fi
\ifnum\sdnuseccenter=0 \titleformat{\section}{\raggedright\Large\bfseries}{\thesection .\quad}{0pt}{}
\fi
	{	
		
		\setlength{\abovecaptionskip}{5pt}
		\setlength{\belowcaptionskip}{5pt}
		\pagenumbering{arabic}
		\setcounter{page}{1}
		\captionsetup{font={small,stretch=1.25}, justification=centering}
		\setlength{\baselineskip}{25pt}
		\section{引言}
	本模板按照《山东师范大学本科生学位论文基本格式要求(2021)》制作。
	
	正文是学位论文的主体部分,其格式规范可根据不同学科的特点和所研究课题的表达需要而定。论文中的计量单位、公式、缩略词和符号等必须遵循国家的有关规定。论文中的图、表应按章(单元)顺序编号,如图1-1为第一章(单元)第1个图,图、表应居中排版。正文中图表标题的标注位置应为图下表上,字号应比正文小一号。
	
	注释与参考文献:注释主要包括释义性注释(对正文中某一内容作进一步解释或补充说明的文字)和引文注释(直接引用他人观点、方法、数据等的文献出处)。注释应以加方括号的数字以上标标出,排印在该页页脚或集中列于文末参考文献表之前。参考文献是作者撰写论文时所引用或参考的已公开发表的文献书目,集中列于文末。参考文献必须是学位申请人真正阅读和参考过的资料,按照学术论文、著作、会议论文、学位论文、电子文献、专利等的顺序用阿拉伯数字统一进行编排。
	
	各学科可根据自身特点选择是否采用注释方式,若不用注释全采用参考文献方式标注的,引用之处应以加方括号的数字以上标标出,按顺序集中列于文末参考文献表,虽未引用但阅读和参考过的资料排在直接引用参考文献之后。
	
	参考文献引用示例\upcite{1, somename}。
	
	参考文献引用示例\cite{1, somename}。
	
\section{论文打印与装订要求}

	论文统一用A4纸标准输出,双面印制,左侧装订。要求纸的四周留足空白边缘,每一面的上方(天头)和左侧(订口)应分别留边25mm,下方(地脚)和右侧(切口)应分别留边20 mm。

\section{一级标题}
	论文中一级标题用三号粗黑体
\subsection{二级标题}
	二级标题用小三号粗黑体
\subsubsection{三级标题}
	三级标题用四号粗黑体。正文内容使用小四号宋体。论文中的所有英文一律采用“Times New Roman”字体。字行间距一般应为固定值25磅。

	The font of English text is Times New Roman. 


\section{页眉与页脚}
	每页要有页眉,其上居中打印“山东师范大学学士学位论文”字样,页码标注在页面底端(页角)外侧。打印请注意确保页码位于页脚外侧,即\textbf{封面、独创声明、目录、摘要请单面打印}。
	
\section{图表的使用}
	论文中的图、表应按章(单元)顺序编号,如图1-1为第一章(单元)第1个图,图、表应居中排版。正文中图表标题的标注位置应为图下表上,字号应比正文小一号。
	
\subsection{图片}
	插图示例:测试图例如图\ref{test-pic}所示,建议使用以htbp为顺序的浮动格式,减少文中“上图”、“下图”之类的描述,转而使用“图X”的描述方式,如必须如前者描述,请仅使用h控制图片浮动。请将文件放置在data/img/文件夹中,图片目录已自动索引,此处不建议也请勿省略后缀名。\par
	\begin{figure}[htbp]
		\centering
		\includegraphics[width=0.5\textwidth]{test.png}
		\caption{示例图片\label{test-pic1}}
	\end{figure}

\subsection{表格}
	请确保\texttt{$\backslash$caption}命令置于\texttt{$\backslash$label}之前,用以保证表头位于表上。

	表格插入示例:测试表格如表\ref{test-table}所示,此处建议使用三线表格,更加复杂的表格(如自动伸缩表格)请使用tabularx或tabu环境,相关宏包已引入。
	\begin{table}[htbp]
		\centering
		\caption{测试表格\label{test-table}}
		\begin{tabular}{ccccc}
			\toprule
			\multirow{2}*{姓名}&\multicolumn{2}{c}{森林}&\multicolumn{2}{c}{房屋}\\
			\cmidrule{2-5}
			&熊大&熊二&光头强&肥波\\
			\midrule
			成绩&97&98&99&70\\
			\bottomrule
		\end{tabular}
	\end{table}

\section{其他特殊环境}
	非分级标题可用\texttt{$\backslash$begin\{enumerate\}}或\texttt{$\backslash$begin\{itemize\}}环境。如:
	
	\begin{itemize}
		\item \textbf{定理:} 定理环境示例:定理\ref{bear-thm}是熊出没定理。
		\begin{theorem}[熊出没定理]\label{bear-thm}
			熊大和熊二以及光头强是好朋友。
		\end{theorem}
		\item \textbf{定义:} 定义环境示例:定义\ref{bear-def}是Bear定义。
		\begin{definition}[Bear定义]\label{bear-def}
			将熊二的弟弟定义为熊三。
		\end{definition}
		\item \textbf{证明:}证明环境示例:
		\begin{proof}[定理\ref{bear-thm}的证明]
			这里是证明环境。
		\end{proof}
		\item \textbf{公式:}
		\begin{enumerate}
			\item \textbf{带编号公式:}编号公式请使用\texttt{$\backslash$begin\{equation\}}环境。
			
			环境公式环境示例:公式\ref{mass-energy equation}为爱因斯坦质能方程。
			\begin{equation}\label{mass-energy equation}
			E=mc^{2}
			\end{equation}
		
			\item \textbf{无编号公式:}无编号公式请使用\texttt{\$\$ $\cdots$ \$\$}环境,句内公式请使用\texttt{\$ $\cdots$ \$}环境。
			
			质能方程$E=mc^{2}$,$E$表示能量,$m$代表质量,而$c$则表示光速(常量,$c=299792.458km/s$)。由阿尔伯特·爱因斯坦提出。$$E=mc^{2}$$该方程主要用来解释核变反应中的质量亏损和计算高能物理中粒子的能量。这也导致了德布罗意波和波动力学的诞生。
		\end{enumerate}
	\end{itemize}

\section{参考文献格式}

\noindent A.连续出版物

	[序号]主要责任者.文献题名[J].刊名,出版年份,卷号(期号):起止页码.

	例:[1]袁庆龙,候文义.Ni-P合金镀层组织形貌及显微硬度研究[J].太原理工大学学报,2001,32(1):51-53.

\noindent B.专著

	[序号]主要责任者.文献题名[M].出版地:出版者,出版年:页码.

	例:[2]刘国钧,郑如斯.中国书的故事[M].北京:中国青年出版社,1979:115.

\noindent C.专著中析出的文献

	[序号]析出责任者.析出题名[A].见(英文用In):专著责任者.书名[M].出版地:出版者,出版年:起止页码.

	例:[3]罗云.安全科学理论体系的发展及趋势探讨[A].见:白春华,何学秋,吴宗之.21世纪安全科学与技术的发展趋势[M].北京:科学出版社,2000:1-5.

\noindent D.学位论文

	[序号]主要责任者.文献题名[D].保存地:保存单位,年份.

	例:[4]张和生.地质力学系统理论[D].太原:太原理工大学,1998.

\noindent E.专利文献

	[序号]专利所有者.专利题名[P].专利国别:专利号,发布日期.

	例:[5]姜锡洲.一种温热外敷药制备方案[P].中国专利:881056078,1983-08-12.

\noindent F.国际、国家标准

	[序号]标准代号.标准名称[S].出版地:出版者,出版年.
	
	例:[6]GB/T 16159—1996.汉语拼音正词法基本规则[S].北京:中国标准出版社,1996.

\noindent G.报纸文章

	[序号]主要责任者.文献题名[N].报纸名,出版年,月(日):版次.

	例:[7]谢希德.创造学习的思路[N].人民日报,1998,12(25):10.

\noindent H.电子文献

	[序号]主要责任者.电子文献题名[文献类型/载体类型].电子文献的出处或可获得网址,发表或更新日期/引用日期(任选).
	
	例:[8]姚伯元.毕业设计(论文)规范化管理与培养学生综合素质[EB/OL].中国高等教育网教学研究(http://www.zhongguogaodengjiaoyuwang.cn/shili.com),2005-2-2.
	}

\titleformat{\section}{\centering\Large\bfseries}{}{0pt}{}[\vspace*{10pt}]
%-----参考文献引入部分,修改参考文献请修改./data/resource/bibliography.tex-----
	{	
		\newpage
\phantomsection
\addcontentsline{toc}{section}{参考文献}
\begin{thebibliography}{99}
	\zihao{5}
	\bibitem{1}
		马尔科姆•沃斯特.现代社会学理论[M].杨善华译.北京:华夏出版社,2000.
	\bibitem{2}
		杰弗里•亚历山大.社会学二十讲:二战以来的理论发展[M].贾春增,董天民,等译.北京:华夏出版社,2000.
	\bibitem{3}
		万俊人.信用伦理及其现代解释[J].孔子研究,2002,(5).
\end{thebibliography}
	}
%-----如果熟悉bibtex,可以注销上面大括号,启用下面大括号代码-------
%	{
%		\bibliographystyle{gbt7714-numerical}
%		\bibliography{./data/resource/reference.bib}  % reference.bib文件路径
%	}

%---附录引入部分,修改附录内容请修改./data/resource/appendix.tex,无需附录请注释\include部分--
	{
		\newpage
\phantomsection
\addcontentsline{toc}{section}{附录}
\section*{附~~录}
\vspace*{-2mm}
\zihao{5}
\setlength{\baselineskip}{25pt}
附录主要列入正文内过分冗长的公式推导、测量问卷、原始数据图表、实验性图片、程序全文及说明等。本项可根据论文需要编排或省略。

这里是附录内容,附录等级为section,若使用下级内容,请在附录内使用subsection*开始下级。

代码环境lstlisting,代码\ref{testcode1}为示例代码1,代码\ref{testcode2}为示例代码2:
\begin{lstlisting}[caption={代码示例1\label{testcode1}}]
int main(int argc,char **argv){
	printf("Hello World!\n");
	return 0;
}
\end{lstlisting}
\begin{lstlisting}[caption={代码示例2\label{testcode2}}]
int main(int argc,char **argv){
	printf("Hello World!\n");
	return 0;
}
\end{lstlisting}
	}

%--致谢部分,修改致谢内容请修改./data/resource/thanks.tex,无需附录请注释\include部分--
	{
	\newpage
\phantomsection
\addcontentsline{toc}{section}{致谢}
\section*{致~~谢}
\vspace*{-2mm}
\zihao{5}
\setlength{\baselineskip}{25pt}

致谢:对论文完成给予帮助的指导老师和同学表示感谢。篇幅不宜冗长,应实事求是,切忌浮夸之词。






\label{thanks}  % 请勿删除此行




	}


\end{document}