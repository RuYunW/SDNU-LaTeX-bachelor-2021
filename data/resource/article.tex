\section{引言}
	本模板按照《山东师范大学本科生学位论文基本格式要求(2021)》制作。
	
	正文是学位论文的主体部分,其格式规范可根据不同学科的特点和所研究课题的表达需要而定。论文中的计量单位、公式、缩略词和符号等必须遵循国家的有关规定。论文中的图、表应按章(单元)顺序编号,如图1-1为第一章(单元)第1个图,图、表应居中排版。正文中图表标题的标注位置应为图下表上,字号应比正文小一号。
	
	注释与参考文献:注释主要包括释义性注释(对正文中某一内容作进一步解释或补充说明的文字)和引文注释(直接引用他人观点、方法、数据等的文献出处)。注释应以加方括号的数字以上标标出,排印在该页页脚或集中列于文末参考文献表之前。参考文献是作者撰写论文时所引用或参考的已公开发表的文献书目,集中列于文末。参考文献必须是学位申请人真正阅读和参考过的资料,按照学术论文、著作、会议论文、学位论文、电子文献、专利等的顺序用阿拉伯数字统一进行编排。
	
	各学科可根据自身特点选择是否采用注释方式,若不用注释全采用参考文献方式标注的,引用之处应以加方括号的数字以上标标出,按顺序集中列于文末参考文献表,虽未引用但阅读和参考过的资料排在直接引用参考文献之后。
	
	参考文献引用示例\upcite{1, somename}。
	
	参考文献引用示例\cite{1, somename}。
	
\section{论文打印与装订要求}

	论文统一用A4纸标准输出,双面印制,左侧装订。要求纸的四周留足空白边缘,每一面的上方(天头)和左侧(订口)应分别留边25mm,下方(地脚)和右侧(切口)应分别留边20 mm。

\section{一级标题}
	论文中一级标题用三号粗黑体
\subsection{二级标题}
	二级标题用小三号粗黑体
\subsubsection{三级标题}
	三级标题用四号粗黑体。正文内容使用小四号宋体。论文中的所有英文一律采用“Times New Roman”字体。字行间距一般应为固定值25磅。

	The font of English text is Times New Roman. 


\section{页眉与页脚}
	每页要有页眉,其上居中打印“山东师范大学学士学位论文”字样,页码标注在页面底端(页角)外侧。打印请注意确保页码位于页脚外侧,即\textbf{封面、独创声明、目录、摘要请单面打印}。
	
\section{图表的使用}
	论文中的图、表应按章(单元)顺序编号,如图1-1为第一章(单元)第1个图,图、表应居中排版。正文中图表标题的标注位置应为图下表上,字号应比正文小一号。
	
\subsection{图片}
	插图示例:测试图例如图\ref{test-pic}所示,建议使用以htbp为顺序的浮动格式,减少文中“上图”、“下图”之类的描述,转而使用“图X”的描述方式,如必须如前者描述,请仅使用h控制图片浮动。请将文件放置在data/img/文件夹中,图片目录已自动索引,此处不建议也请勿省略后缀名。\par
	\begin{figure}[htbp]
		\centering
		\includegraphics[width=0.5\textwidth]{test.png}
		\caption{示例图片\label{test-pic1}}
	\end{figure}

\subsection{表格}
	请确保\texttt{$\backslash$caption}命令置于\texttt{$\backslash$label}之前,用以保证表头位于表上。

	表格插入示例:测试表格如表\ref{test-table}所示,此处建议使用三线表格,更加复杂的表格(如自动伸缩表格)请使用tabularx或tabu环境,相关宏包已引入。
	\begin{table}[htbp]
		\centering
		\caption{测试表格\label{test-table}}
		\begin{tabular}{ccccc}
			\toprule
			\multirow{2}*{姓名}&\multicolumn{2}{c}{森林}&\multicolumn{2}{c}{房屋}\\
			\cmidrule{2-5}
			&熊大&熊二&光头强&肥波\\
			\midrule
			成绩&97&98&99&70\\
			\bottomrule
		\end{tabular}
	\end{table}

\section{其他特殊环境}
	非分级标题可用\texttt{$\backslash$begin\{enumerate\}}或\texttt{$\backslash$begin\{itemize\}}环境。如:
	
	\begin{itemize}
		\item \textbf{定理:} 定理环境示例:定理\ref{bear-thm}是熊出没定理。
		\begin{theorem}[熊出没定理]\label{bear-thm}
			熊大和熊二以及光头强是好朋友。
		\end{theorem}
		\item \textbf{定义:} 定义环境示例:定义\ref{bear-def}是Bear定义。
		\begin{definition}[Bear定义]\label{bear-def}
			将熊二的弟弟定义为熊三。
		\end{definition}
		\item \textbf{证明:}证明环境示例:
		\begin{proof}[定理\ref{bear-thm}的证明]
			这里是证明环境。
		\end{proof}
		\item \textbf{公式:}
		\begin{enumerate}
			\item \textbf{带编号公式:}编号公式请使用\texttt{$\backslash$begin\{equation\}}环境。
			
			环境公式环境示例:公式\ref{mass-energy equation}为爱因斯坦质能方程。
			\begin{equation}\label{mass-energy equation}
			E=mc^{2}
			\end{equation}
		
			\item \textbf{无编号公式:}无编号公式请使用\texttt{\$\$ $\cdots$ \$\$}环境,句内公式请使用\texttt{\$ $\cdots$ \$}环境。
			
			质能方程$E=mc^{2}$,$E$表示能量,$m$代表质量,而$c$则表示光速(常量,$c=299792.458km/s$)。由阿尔伯特·爱因斯坦提出。$$E=mc^{2}$$该方程主要用来解释核变反应中的质量亏损和计算高能物理中粒子的能量。这也导致了德布罗意波和波动力学的诞生。
		\end{enumerate}
	\end{itemize}

\section{参考文献格式}

\noindent A.连续出版物

	[序号]主要责任者.文献题名[J].刊名,出版年份,卷号(期号):起止页码.

	例:[1]袁庆龙,候文义.Ni-P合金镀层组织形貌及显微硬度研究[J].太原理工大学学报,2001,32(1):51-53.

\noindent B.专著

	[序号]主要责任者.文献题名[M].出版地:出版者,出版年:页码.

	例:[2]刘国钧,郑如斯.中国书的故事[M].北京:中国青年出版社,1979:115.

\noindent C.专著中析出的文献

	[序号]析出责任者.析出题名[A].见(英文用In):专著责任者.书名[M].出版地:出版者,出版年:起止页码.

	例:[3]罗云.安全科学理论体系的发展及趋势探讨[A].见:白春华,何学秋,吴宗之.21世纪安全科学与技术的发展趋势[M].北京:科学出版社,2000:1-5.

\noindent D.学位论文

	[序号]主要责任者.文献题名[D].保存地:保存单位,年份.

	例:[4]张和生.地质力学系统理论[D].太原:太原理工大学,1998.

\noindent E.专利文献

	[序号]专利所有者.专利题名[P].专利国别:专利号,发布日期.

	例:[5]姜锡洲.一种温热外敷药制备方案[P].中国专利:881056078,1983-08-12.

\noindent F.国际、国家标准

	[序号]标准代号.标准名称[S].出版地:出版者,出版年.
	
	例:[6]GB/T 16159—1996.汉语拼音正词法基本规则[S].北京:中国标准出版社,1996.

\noindent G.报纸文章

	[序号]主要责任者.文献题名[N].报纸名,出版年,月(日):版次.

	例:[7]谢希德.创造学习的思路[N].人民日报,1998,12(25):10.

\noindent H.电子文献

	[序号]主要责任者.电子文献题名[文献类型/载体类型].电子文献的出处或可获得网址,发表或更新日期/引用日期(任选).
	
	例:[8]姚伯元.毕业设计(论文)规范化管理与培养学生综合素质[EB/OL].中国高等教育网教学研究(http://www.zhongguogaodengjiaoyuwang.cn/shili.com),2005-2-2.