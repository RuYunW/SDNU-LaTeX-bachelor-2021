% {
% 	\phantomsection
% 	\addcontentsline{toc}{section}{中文摘要}
% 	\centering
% 	\noindent\textbf{\zihao{3}\sdnutitlechs}\\
% 	\vspace*{5mm}
% 	\textbf{\zihao{4}摘~要}\\
% 	\vspace*{5mm}
% }
% \setlength{\baselineskip}{25pt}
% \abstractchs
% \par
% \vspace*{5mm}
% \noindent\textbf{关键词:}\sdnukeywchs
% \newpage
% {
% 	\phantomsection
% 	\addcontentsline{toc}{section}{英文摘要}
% 	\centering
% 	\noindent\textbf{\zihao{3}\sdnutitleen}\\
% 	\vspace*{5mm}
% 	\textbf{\zihao{4}ABSTRACT}\\
% 	\vspace*{5mm}
% }
% \setlength{\baselineskip}{25pt}
% \abstracten
% \par
% \vspace*{5mm}
% \noindent\textbf{KEY WORDS: }\sdnukeywen
% \newpage


\clearpage
\setcounter{page}{1}
\begin{center}
\fontsize{15.75pt}{\baselineskip}
\vspace*{24pt}
\heiti{\sdnutitlechs}  % 题目-黑体三号

\vspace*{14pt}
\fontsize{14pt}{\baselineskip}
\fangsong{\sdnuauthorchs} % 作者-仿宋四号

% \vspace*{20pt}
\fontsize{10.5pt}{20pt}
\fangsong{(山东师范大学\sdnucollegechs\sdnumajorchs\sdnuclasschs )} % 学校-仿宋五号
\vspace*{18pt}

\end{center}

% 中文摘要
\phantomsection
\addcontentsline{toc}{section}{摘要}
% \section*{摘要}
% \vspace*{-2mm}

% \setlength{\baselineskip}{25pt}
\fontsize{12pt}{25pt}
\textbf{摘要:}中文摘要与关键词:对毕业论文(设计)核心内容的集中陈述,简要说明研究目的、研究方法、主要结果(结论),150-300字,中文摘要后需列出3-5个关键词。(中英文摘要合起来尽量独占一页,如果内容较多也可占据两页)

\vspace*{14pt}
\textbf{关键词:}\sdnukeywchs

\vspace*{4pt}
\fontsize{10.5pt}{25pt}
中图分类号:\textrm{\classification}
\vspace*{18pt}

% 英文摘要
% \newpage
\phantomsection
\addcontentsline{toc}{section}{Abstract}
% \section*{Abstract}
% \vspace*{-2mm}

\begin{center}
\fontsize{15.75pt}{\baselineskip}
\vspace*{24pt}
\textrm{\textbf{\sdnutitleen}}  % 题目-新罗马三号

\fontsize{12pt}{\baselineskip}
\vspace*{12pt}
\textrm{\sdnuauthorens} % 作者-新罗马小四

\textrm{(\sdnucollegeen, \sdnuen)} % 单位-新罗马小四

\end{center}

\fontsize{12pt}{25pt}
\textbf{Abstract:~}The content of the English abstract should be basically the same as that of the Chinese abstract, and the English keywords corresponding to the Chinese keywords should be listed after the English abstract.

\vspace*{6pt}
\textbf{Key words:~}\sdnukeywen
